% (c) 2002 Matthew Boedicker <mboedick@mboedick.org> (original author) http://mboedick.org
% (c) 2003-2007 David J. Grant <davidgrant-at-gmail.com> http://www.davidgrant.ca
% (c) 2008 Nathaniel Johnston <nathaniel@nathanieljohnston.com> http://www.nathanieljohnston.com
% (l) 2012 Arun I B <arunib@smail.iitm.ac.in> http://www.ee.iitm.ac.in/~ee10s026/
%This work is licensed under the Creative Commons Attribution-Noncommercial-Share Alike 2.5 License. To view a copy of this license, visit http://creativecommons.org/licenses/by-nc-sa/2.5/ or send a letter to Creative Commons, 543 Howard Street, 5th Floor, San Francisco, California, 94105, USA.

\documentclass[letterpaper,11pt]{article}
\usepackage[UTF8]{ctex}
\newlength{\outerbordwidth}
\pagestyle{empty}
\raggedbottom
\raggedright
\usepackage[svgnames]{xcolor}
\usepackage{framed}
\usepackage{times}
\usepackage{tocloft}
\usepackage{graphicx}
\usepackage{multirow}
\usepackage[utf8]{inputenc}
\usepackage{tabularx}
\title{Aparna-CV}
%-----------------------------------------------------------
%Edit these values as you see fit

\setlength{\outerbordwidth}{3pt}  % Width of border outside of title bars
\definecolor{shadecolor}{gray}{0.75}  % Outer background color of title bars (0 = black, 1 = white)
\definecolor{shadecolorB}{gray}{0.93}  % Inner background color of title bars


%-----------------------------------------------------------
%Margin setup

\setlength{\evensidemargin}{-0.25in}
\setlength{\headheight}{0in}
\setlength{\headsep}{0in}
\setlength{\oddsidemargin}{-0.25in}
\setlength{\paperheight}{11in}
\setlength{\paperwidth}{8.5in}
\setlength{\parindent}{24pt}
\setlength{\tabcolsep}{0in}
\setlength{\textheight}{9.5in}
\setlength{\textwidth}{7in}
\setlength{\topmargin}{-0.3in}
\setlength{\topskip}{0in}
\setlength{\voffset}{0.1in}


%-----------------------------------------------------------
%Custom commands
\newcommand{\resitem}[1]{\item #1 \vspace{-2pt}}
\newcommand{\resheading}[1]{\vspace{8pt}
  \parbox{\textwidth}{\setlength{\FrameSep}{\outerbordwidth}
    \begin{shaded}
\setlength{\fboxsep}{0pt}\framebox[\textwidth][l]{\setlength{\fboxsep}{4pt}\fcolorbox{shadecolorB}{shadecolorB}{\textbf{\sffamily{\mbox{~}\makebox[6.762in][l]{\large #1} \vphantom{p\^{E}}}}}}
    \end{shaded}
  }\vspace{-5pt}
}
\newcommand{\ressubheading}[4]{
\begin{tabular*}{6.5in}{l@{\cftdotfill{\cftsecdotsep}\extracolsep{\fill}}r}
		\textbf{#1} & #2 \\
		\textit{#3} & \textit{#4} \\
\end{tabular*}\vspace{-6pt}}
%-----------------------------------------------------------


\begin{document}

%-----------------------------------------------------------
%Insert IIT Madras Logo 
\begin{tabular*}{7in}{l@{\extracolsep{\fill}}r}
%  & \multirow{4}{*}{\includegraphics[scale=0.19]{iitmlogo}}\\
  & \\
%-----------------------------------------------------------  
  \textbf{\emph{\Large 个人资料} } & \\
  \indent {姓名:郑鹏} & \\
  \indent {性别:男} & \\
  \indent {学历:本科}  \\
  \indent {手机号码:15732115701} \\
  \indent {e-mail:15732115701@163.com} \\
\end{tabular*}
\\


%%%%%%%%%%%%%%%%%%%%%%%%%%%%%%
\resheading{求职意向}
%%%%%%%%%%%%%%%%%%%%%%%%%%%%%%
\begin{itemize}
\item
	\indent {计算机视觉工程师实习生}	\\

\end{itemize}

%%%%%%%%%%%%%%%%%%%%%%%%%%%%%%
\resheading{个人技能}
%%%%%%%%%%%%%%%%%%%%%%%%%%%%%%
\begin{itemize}
\item
	使用语言
	\begin{itemize}
		\resitem{{\bf } Python, MATLAB}
	\end{itemize}

\item
	计算机视觉相关
	\begin{itemize}
		\resitem{熟练使用 OpenCV 处理图像、视频等}
		\resitem{熟练使用 MATLAB 处理数字图像}
		\resitem{较熟练使用 skimage 等其他开源图像处理库处理图像}
		\resitem{熟练使用 OpenCV 处理图像、视频等}
	\end{itemize}

\item
	算法实现原理相关
	\begin{itemize}
		\resitem{熟练掌握基本图像操作的实现原理,例如 Canny边缘提取、Hough变换、双边滤波等}
		\resitem{较熟练掌握《Learning OpenCV》、《数字图像处理》(冈萨雷斯)等书中内容}
		\resitem{了解常用的机器学习涉及的视觉领域常用算法,例如:CNN、GAN 等}
		\resitem{多次参加数学建模竞赛,熟练掌握相关建模算法}
	\end{itemize}
\item
	英语相关
	\begin{itemize}
		\resitem{四级 597}
		\resitem{六级 526}
		\resitem{大学生英语竞赛二等奖}
	\end{itemize}

\end{itemize}

\vspace{100mm}

%%%%%%%%%%%%%%%%%%%%%%%%%%%%%%
\resheading{相关经历}
%%%%%%%%%%%%%%%%%%%%%%%%%%%%%%
\begin{itemize}
\item
	正在参与所识南京邮电大学教授与 KTH 合作的计算机视觉项目,主要负责叶脉提取部分,目前已基本完成

\item
	在 Coursera 等在线学习平台,高分通过多门课程

\item
	利用 Python+OpenCV 自行完成简单AR游戏等小成果

\end{itemize}


%%%%%%%%%%%%%%%%%%%%%%%%%%%%%%
\resheading{在校情况和比赛经历}
%%%%%%%%%%%%%%%%%%%%%%%%%%%%%%
\begin{itemize}
\item
	大一、大二年级排名: 16/384, 9/384, 14/384, 15/384
	
\item
	全国大学生英语竞赛二等奖

\item
	蓝桥杯省二等奖

\item
	全国大学生数学建模省二等奖


\end{itemize}


%%%%%%%%%%%%%%%%%%%%%%%%%%%%%%
\resheading{教育经历}
%%%%%%%%%%%%%%%%%%%%%%%%%%%%%%
\begin{itemize}
\item
	2015-2019 河北师范大学 软件工程

\end{itemize}


%%%%%%%%%%%%%%%%%%%%%%%%%%%%%%
\resheading{自我评价}
%%%%%%%%%%%%%%%%%%%%%%%%%%%%%%
\begin{itemize}
\item
	热爱计算机视觉和机器学习,求知欲强,不怕吃苦,抗压能力强;大一大二两年坚持6:20起床晨读,坚持不懈;数理基础功底扎实,乐于钻研;乐于帮助他人,善于合作交流

\end{itemize}

\end{document}
